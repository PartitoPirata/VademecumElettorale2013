\documentclass[paper=a4,11pt]{scrartcl}
\usepackage[italian]{babel}
\usepackage[utf8]{inputenc}

\usepackage[T1]{fontenc}
\usepackage{lmodern}

\usepackage{graphicx}
\usepackage{enumerate}

\usepackage[hidelinks]{hyperref}

\usepackage{framed}

\titlehead{\centering\includegraphics[width=\columnwidth]{bebaslogo}}
% la linea sopra aggiunge l'immagine sulla copertina.

\title{VADEMECUM Raccolta Firme\\ Elezioni Politiche 2013}
\author{GdL Elettorale del Partito Pirata\\<\href{mailto:elettorale@lists.partito-pirata.it}{elettorale@lists.partito-pirata.it}>}
\date{29 dicembre 2012}

\begin{document}
\maketitle
\begin{abstract}
Car* tutt*,

a prescindere da ciò che faremo sul serio, a prescindere dall'effettiva 
praticabilità di questa opzione politica, ecco a voi un VADEMECUM per la 
raccolta firme per le Elezioni Politiche 2013. Si tratta di una 'sintesi' 
ragionata di quanto prodotto - sul terreno tecnico elettorale - in questi 120 
giorni di lavoro del gdl elettorale. Spero di aver fatto cosa gradita. 

Qualora si ravvisassero imprecisioni o sbavature, prego segnalare.
\end{abstract}

\tableofcontents

\section{Premessa}
Chi raccoglie le firme per la presentazione della lista “Partito Pirata 
Italiano” esercita un diritto riconosciuto dalla legge. 

L'obiettivo è raccogliere il numero di firme di sottoscrittori prescritto per 
la lista per le circoscrizioni elettorali per la Camera dei deputati e per la 
circoscrizone regionale per il Senato della Repubblica.

Il numero complessivo delle firme è ridotto ad un quarto per via della 
conclusione anticipata della legislatura e il conseguente decreto ministeriale 
approvato alle camere.

In particolare, per la presentazione per ciascuna lista, occorrono i seguenti 
sottoscrittori:

\begin{itemize}
\item Alla Camera dei deputati, la presentazione delle liste di candidati deve 
essere sottoscritta:
\begin{enumerate}[a)]
\item da almeno 1.500 e da non più di 2.000 elettori iscritti nelle liste 
elettorali di comuni compresi nelle circoscrizioni fino a 500.000 abitanti;
\item da almeno 2.500 e da non più di 3.000 elettori iscritti nelle liste 
elettorali di comuni compresi nelle circoscrizioni con più di 500.000 e fino a 
1.000.000 di abitanti;
\item da almeno 4.000 e da non più di 4.500 elettori iscritti nelle liste 
elettorali di comuni compresi nelle circoscrizioni con più di 1.000.000 di 
abitanti.\end{enumerate}

\item Al Senato della Repubblica, la dichiarazione di presentazione di ciascuna 
lista deve essere sottoscritta:
\begin{enumerate}[a)]
\item da almeno 1.000 e da non più di 1.500 elettori iscritti nelle liste 
elettorali di comuni compresi nelle regioni fino a 500.000 abitanti;
\item da almeno 1.750 e da non più di 2.500 elettori iscritti nelle liste 
elettorali di comuni compresi nelle regioni con più di 500.000 abitanti e fino 
a 1.000.000 di abitanti;
\item da almeno 3.500 e da non più di 5.000 elettori iscritti nelle liste 
elettorali di comuni compresi nelle regioni con più di 1.000.000 di abitanti.\end{enumerate}

\item Nel collegio uninominale della Valle d’Aosta, sia per l’elezione della Camera 
quanto per quella del Senato, la dichiarazione di presentazione delle 
candidature deve essere sottoscritta da almeno 300 e da non più di 600 elettori 
del collegio.

\item Nella regione Trentino-Alto Adige, per l’elezione del Senato della Repubblica, 
la dichiarazione di presentazione del gruppo di candidati deve essere 
sottoscritta da almeno 1.750 e da non più di 2.500 elettori iscritti nelle 
liste elettorali di comuni della regione, mentre la dichiarazione di 
presentazione della candidatura individuale nei collegi uninominali deve essere 
sottoscritta da almeno 1.000 e da non più di 1.500 elettori del collegio.
\end{itemize}

\textbf{NOTA BENE}: tutte queste cifre sono da intendersi ridotte di 3/4 in seguito al 
decreto ministeriale per le elezioni 2013

\begin{leftbar}\textbf{ATTENZIONE!} Tutte le firme devono essere prima autenticate, all'atto della 
sottoscrizione, e successivamente certificate presso il Comune di residenza 
dell'elettore sottoscrittore. Inoltre, è consigliabile (ma non obbligatorio) 
raccogliere il numero massimo di firme richiesto per legge, perché alcune 
potrebbero essere annullate o riportate erroneamente.\end{leftbar}

\section{Moduli}
Le sottoscrizioni degli elettori a corredo delle liste per camera e senato 
devono essere contenute in appositi moduli, recanti in ciascun foglio il 
contrassegno di lista, il cognome, il nome, il luogo e la data di nascita di 
ciascun candidato, nonché il cognome, il nome, il luogo e la data di nascita di 
ognuno dei sottoscrittori ed il comune di iscrizione nelle liste elettorali.

(NB: MODULO bozza POSTATO IN GDL ELETTORALE!)\footnote{\href{http://lists.partito-pirata.it/pipermail/elettorale/2012-December/000359.html}{[Elettorale \#360] Moduli Raccolta Firme.}}

\subsection{Come si compongono i moduli?}
I moduli per la raccolta firma sono complessivamente quattro: due per la 
raccolta firme della lista camera e due per la raccolta firme della lista 
senato, che vanno impiegati di concerto, usando parte di uno e parte dell'altro 
modulo per ciascun tipo di raccolta. Tutto ciò tornerà utile per il reperimento 
dei certificati elettorali, in forma agevolata
\begin{itemize}
\item Moduli per la lista alla CAMERA:
	\begin{enumerate}[1)]
	\item si raccolgono le firme degli iscritti nei comuni capoluoghi di provincia;
	\item si raccolgono le firme dei cittadini iscritti nei comuni NON capoluogo di 
	provincia.\end{enumerate}
\item Moduli per la lista al SENATO:
	\begin{enumerate}[1)]
	\item si raccolgono le firme dei cittadini elettori (over 25enni!) nei comuni 
	capoluoghi di provincia
	\item si raccolgono le firme dei cittadini elettori (over 25enni!) nei comuni NON 
	capoluoghi di provincia.\end{enumerate}
\end{itemize}
Trattasi di buona prassi organizzativa, non obbligatoria. L'esperienza ci 
suggerisce di adottare questo piccolo accorgimento.


\begin{leftbar}\textbf{ATTENZIONE!} Ciascun blocco di moduli dovrà essere accompagnato dalla lista dei candidati 
per il relativo collegio elettorale e per la camera e per il senato. Non è 
possibile raccogliere firme di sottoscrittori in assenza della suddetta lista.\end{leftbar}

\section{Autenticatori}
Gli autenticatori/trici devono essere presenti al banchetto firma durante le 
operazioni di raccolta delle sottoscrizioni.

\subsection{Chi può svolgere il ruolo di autenticatore/trice?}
Gli autenticatori/trici abilitati, in base alla legge 28 aprile 1998, n. 130 e 
all'art. 4 della legge 30 aprile
1999, n. 120 sono:
\begin{itemize}
\item Notai;
\item Giudici di pace;
\item Segretari delle Procure della Repubblica;
\item Cancellieri e collaboratori delle cancellerie dei Tribunali o primo 
dirigente o dirigente superiore
della cancelleria dell'ufficio giudiziario ossia Corte d'Appello, Tribunale o 
Pretura;
\item Presidenti delle Province;
\item Assessori provinciali;
\item Presidenti di Consigli Provinciali;
\item Segretari provinciali;
\item Funzionari incaricati dal Presidente della Provincia;
\item Consiglieri Provinciali che comunichino la propria disponibilità al 
Presidente della
\item Provincia;
\item Sindaci;
\item Assessori comunali;
\item Presidenti di Consigli Comunali;
\item Segretari comunali;
\item Funzionari incaricati dal Sindaco;
\item Consiglieri Comunali che comunichino la propria disponibilità al Sindaco;
\item Presidenti dei Consigli Circoscrizionali;
\item Vice Presidenti dei Consigli Circoscrizionali.
\end{itemize}

\subsection{Quali sono gli/le autenticatori/trici che per svolgere il loro ruolo devono 
comunicare la loro disponibilità o chiedere l’autorizzazione a qualcuno?}
\begin{enumerate}
\item Comune\begin{itemize}
\item I consiglieri comunali devono prima comunicare la loro disponibilità al 
sindaco. Si tratta solo di una comunicazione che non comporta autorizzazione;
\item i funzionari comunali devono essere autorizzati dal Sindaco.\end{itemize}

\item Provincia
\begin{itemize}
\item Gli assessori comunali non hanno bisogno di nessuna autorizzazione;
\item i consiglieri provinciali devono prima comunicare la loro disponibilità al 
presidente della provincia. Si tratta solo di una comunicazione che non 
comporta autorizzazione;
\item i funzionari provinciali devono essere autorizzati dal Presidente della 
Provincia;
\item Gli assessori provinciali non hanno bisogno di nessuna autorizzazione.
\end{itemize}
\item Dipendenti del Ministero della Giustizia
\begin{itemize}
\item I Cancellieri possono autenticare le firme all'interno dei loro uffici in 
orario di lavoro, per uscire fuori dagli uffici fuori orario di lavoro (per 
autenticare, per esempio ai tavoli, devono essere autorizzati dal presidente 
del tribunale o della Corte di Appello);
\item i Giudici di Pace, per poter autenticare le firme, devono essere autorizzati 
dal coordinatore dei giudici di pace.\end{itemize}\end{enumerate}

\subsection{Chi sono i funzionari comunali o provinciali?}
In base alla legge 28 aprile 1998, n. 130 possono autenticare le firme ``i 
funzionari incaricati dal sindaco e dal presidente della provincia.'' Tuttavia, 
il termine funzionario viene variamente interpretato: è prassi in alcuni Comuni 
considerare funzionari solo i dipendenti dal $6^{\circ}$ livello compreso ed in su; in 
altri Comuni, invece, si considerano funzionari tutti i dipendenti. 
Nell'incertezza interpretativa la tesi che conviene sostenere è quella a noi 
favorevole, ossia che per funzionario si intenda qualsiasi dipendente.

Tuttavia, questa interpretazione non può essere imposta.

\subsection{In quale ambito territoriale possono operare gli/le autenticatori/trici?}
In base alla circolare del Ministero degli Interni n. 158/99, che interpreta 
l'art. 14 della legge 53 del 1990 che a sua volta ha modificato la legge 352 
del 1970, ogni autenticatore/trice può autenticare le firme di tutti i 
cittadini italiani purché lo faccia all'interno del territorio di sua 
competenza.

\begin{leftbar}\textbf{ATTENZIONE!} Se il vostro autenticatore è, in ipotesi, un consigliere comunale 
questi potrà autenticare solo le firme dei residenti nel relativo Comune; 
viceversa se il vostro autenticatore è un consigliere provinciale questi potrà 
autenticare tutti i firmatari della provincia.\end{leftbar}

\section{Raccogliere le firme}
\subsection{Chi può firmare? Accertamenti prima della sottoscrizione.}
Possono firmare solo i cittadini italiani con diritto di voto e residenti in 
Italia. Va richiesta, anzitutto, la residenza del sottoscrittore ed in base a 
quest'ultima decidere su quale modulo far firmare il sottoscrittore (ad es. uno 
per Napoli e un altro per Giugliano). In questo modo si può agevolare, come 
detto, la fase successiva, quella di certificazione delle firme. Infatti, se le 
firme su uno stesso modulo sono tutte di sottoscrittori residenti nello stesso 
comune, occorrerà recarsi con quel modulo solo in quel determinato Comune. In 
secondo luogo, va richiesto il documento d’identità del sottoscrittore che 
serve per l’autenticatore/trice ai fini dell'identificazione e soprattutto ad 
evitare errori nella compilazione del modulo.

\subsection{Quando è possibile procedere con la raccolta delle firme?}
La legge non stabilisce un dies a quo, ma si limita a prevedere la nullità 
delle autenticazioni anteriori al $180^{\circ}$ giorno precedente il termine finale 
fissato per la presentazione delle candidature (art. 14, comma 3, della legge 
21 marzo 1990, n. 53, e successive modificazioni). Inutile dire che il PP-it è 
ABBONDANTEMENTE (...) dentro questi tempi.

\subsection{Cosa bisogna scrivere sul modulo?}
Dietro presentazione di un documento d’identità, bisogna scrivere a 
STAMPATELLO, in modo chiaro e
semplice, sulle rispettive colonne e righe del modulo, i seguenti dati del 
sottoscrittore:\begin{itemize}
\item nome e cognome;
\item luogo e data di nascita;
\item Comune di iscrizione nelle liste elettorali;
\item estremi del documento di identità;
\item firma.\end{itemize}

\begin{leftbar}\textbf{ATTENZIONE!} Bisogna prestare attenzione ai casi particolari, come ad esempio i 
doppi nomi, che vanno indicati.\end{leftbar}

\subsection{Quanti moduli può firmare il sottoscrittore?}
È ammessa la sottoscrizione sia del modulo della lista per la CAMERA sia di 
quella per il SENATO (NO under 25 per il senato), ma nessun elettore può 
sottoscrivere dichiarazioni di presentazione di lista appartenenti a
soggetti politici differenti. Ovvero: ogni elettore può sottoscrivere le liste 
di un solo partito!

\begin{leftbar}\textbf{ATTENZIONE!} Ricordate al sottoscrittore che non potrà più sottoscrivere altre raccolte 
firme!\end{leftbar}

\subsection{Data di nascita del sottoscrittore}
ELETTORATO ATTIVO: Bisogna verificare sempre che il sottoscrittore abbia già 
compiuto 18 anni (25 per il senato!)

\subsection{Comune di iscrizione nelle liste elettorali}
Quando si legge l’indirizzo sul documento (ok patente, ok CI) è bene chiedere 
sempre se la residenza sia cambiata.

\emph{Buona regola per la compilazione: sul modulo bisogna riportare sempre quanto richiesto e bisogna essere sempre molto CHIARI E PRECISI nella compilazione.}

\section{Certificazione elettorale}
Ogni firma raccolta va certificata presso il Comune di residenza del 
firmatario. L'art. 13 bis della legge
regionale 20 Marzo 1951, n. 29 prevede che nei 20 giorni precedenti la 
presentazione della lista, tutti i Comuni assicurino agli elettori la 
possibilità di ottenere la certificazione dell’iscrizione nelle liste 
elettorali e di sottoscrivere celermente
le liste, per non meno di dieci ore al giorno dal lunedì al venerdì, otto 
ore il sabato e la domenica, svolgendo tale funzione anche in proprietà 
comunali diverse dalla residenza municipale.
Gli orari sono resi noti al pubblico mediante loro esposizione chiaramente 
visibile anche nelle ore di chiusura degli uffici.E' buona abitudine verificare 
presso i propri comuni l'applicazione di questa legge (anzitempo!). I comuni 
'virtuosi' dovrebbero aver già provveduto a tali oneri.

\subsection{Cos’è la certificazione?}
Per ogni firmatario va certificata l’iscrizione alle liste elettorali da parte 
del Comune di competenza.
Questo vuol dire che il modulo va consegnato all’ufficio elettorale del comune 
di residenza dei firmatari.
Se l’ufficio elettorale di un Comune riscontra che il firmatario è 
effettivamente elettore di quel Comune riporta il numero di iscrizione alle 
liste elettorali del firmatario di fianco alla firma nell’apposito spazio e 
compila, firma e timbra il modulo.

\subsection{Come può essere fatta la certificazione?}
La certificazione può essere fatta in due modi: singola o collettiva. La 
certificazione ``singola'' consiste nell'allegare i singoli certificati 
elettorali dei firmatari: se 20 elettori hanno firmato occorrono 20 certificati 
elettorali. La certificazione ``collettiva,'' molto più semplice e ragionevole, 
si fa apponendo il numero di iscrizione nelle liste elettorali nell'apposito 
spazio a fianco di ciascuna firma.

\subsection{Cosa succede se si scopre all’atto della certificazione che solo una parte 
delle firme presenti in un modulo sono di sottoscrittori residenti nel comune 
in cui viene presentato il modulo?}

Può succedere che uno dei firmatari non risulti residente in quel comune. Se l’
ufficio elettorale è disponibile può riferirvi dove si è trasferita la persona 
interessata, ma non è tenuto a farlo. Nel caso vi comunichino il nuovo 
indirizzo si può procedere con la certificazione nel nuovo Comune di residenza.

\subsection{Cosa succede se una firma non può essere certificata?}
Nel caso in cui non fosse possibile certificare una firma, la stessa verrà 
annullata, ma SENZA inficiare tutto il modulo.

\subsection{Cosa bisogna fare all’atto del ritiro delle firme certificate?}
All’atto del ritiro delle firme certificate occorre, come sempre, controllare 
che lo spazio per la certificazione sia stato compilato in tutte le sue parti e 
ci siano tutti i timbri. Inoltre, occorre accertare che la data della 
certificazione non sia antecedente alla data dell’autenticazione.
Se la certificazione non viene apposta le firme non valgono; e se non vengono 
certificate alcune firme, saranno solo queste prive di valore. Nel caso di 
certificati a parte, è estremamente opportuno che siano spillati al modulo cui 
si riferiscono.

\textbf{SUGGERIMENTO}: In questo tipo di 'lavoro' è meglio essere 'pedanti' nei 
confronti dei funzionari amministrativi, piuttosto che correre il rischio di 
inficiare l'intero lavoro assecondando 'lassismi fiduciari'!

\begin{leftbar}\textbf{ATTENZIONE!} Verificate che sia apposto il timbro a metà tra i fogli.\end{leftbar}

\subsection{Quando bisogna avviare la fase di certificazione delle firme?}
E' importante procedere nella certificazione un po’ per volta, di modo da 
poter verificare di eventuali errori da far correggere dal certificatore/trice 
o autenticatore/trice.

\subsection{Cosa si fa dei moduli certificati?}
I moduli certificati devono essere allegati alla presentazione delle liste!!!

\subsubsection{Errori possibili}
Può accadere che i soggetti abilitati all'autenticazione e alla certificazione 
elettorale commettano degli errori. Niente paura! Qualsiasi errore può essere 
sanato apponendo a fianco della correzione il timbro tondo dell'Ufficio e la 
firma del funzionario.

\section{Organizzazione della raccolta}
Procuratevi uno o più tavoli pieghevoli e trasportabili, sufficientemente 
grandi da permettere a più persone di scrivere contemporaneamente. Il tavolo 
deve essere adeguatamente 'addobbato' per rendere visibile cosa si sta facendo 
anche da lontano. Curare, dunque, l'aspetto comunicativo.
Non aspettate che le persone vengano da voi, non succede! Organizzate il 
banchetto in una piazza molto frequentata della vostra città, se possibile 
anche davanti al Comune. Ben vengano presenze negli immediati paraggi di 
'kermesse' molto partecipate. Quando saprete dove sarà allestito il tavolo di 
raccolta firme segnalate il banchetto attraverso internet o quant'altro torni 
utile a reclamizzarvi. L’elenco dei banchetti dev'essere pubblicato e 
aggiornato per consentire agli elettori sottoscrittori di poter apporre la 
firma. E' buona abitudine porre sul banchetto - magari sotto gazebo (3x2) con 
bandiere - una cassettina per le sottoscrizioni politiche volontarie, di solito 
funziona! E' evidente che non guasta affatto avere dei nostri volantini - con 
tanto di riferimenti locali - con cui si spiega cos'è il Partito Pirata ecc..
ecc..ecc...

\subsection{Voglio organizzare un tavolo: cosa devo fare?}
\subsubsection{Occupazione del suolo pubblico}
Per installare un tavolo in una piazza o in una strada, occorre chiedere al 
Comune l'Autorizzazione per l'Occupazione di Suolo Pubblico, indicando nella 
richiesta il giorno, l'ora, la superficie di suolo occupata con il tavolo, 
tenendo presente che, in base alla legge n. 549 del 28 dicembre 1995, se lo
spazio occupato è inferiore ai 10 metri quadrati, NON si paga la relativa 
tassa (Legge n. 549 del 28 dicembre 1995 - art. 3 comma 67: ``Sono esonerati 
dall'obbligo al pagamento della tassa per l'occupazione di spazi ed aree 
pubbliche coloro i quali promuovono manifestazioni od iniziative a carattere 
politico, purché l'area occupata non ecceda i 10 metri quadrati.'').
 Tuttavia, alcuni Comuni richiedono il pagamento mediante marche da bollo.
Una volta ricevuta l’autorizzazione, è opportuno comunicare alla Polizia 
(ufficio Digos) o alla locale stazione dei Carabinieri che in quella data si 
effettuerà la raccolta firme. L'autorizzazione deve essere portata al tavolo 
perché può essere chiesta dai Vigili Urbani. Per non gravare sull'organizzazione 
è preferibile chiedere un'unica autorizzazione CUMULATIVA ovvero contenente 
tutte le date e le fasce d'orario con tanto di ubicazioni per l'intera fase di 
raccolta firme.

\subsubsection{Quanti pirati occorrono?}
Il numero minimo, per una buona riuscita, è 5: l'autenticatore + due pirati 
'segretari' (che compilano i moduli), altri due torneranno utili per chiedere 
ai passanti di sottoscrivere i moduli (magari facendo anche volantinaggio). Le 
'braccia' torneranno utili anche per l'allestimento 'materiale' del presidio 
(gazebo + tavolini + sedie + moduli + volantini + cassettina + materiale di 
cancelleria + bandiere). Se il numero di pirati diminuisce, aumenta la 
complessità delle operazioni. 

\section{NOTA A MARGINE}
La presentazione delle liste dei candidati è effettuata presso le cancellerie 
delle Corti di Appello o dei Tribunali dalla ore 8 del trentacinquesimo alle 
ore 20 del trentaquattresimo giorno antecedente quello della votazione!!! Anche 
qui è preferibile prendere contatto con i funzionari con largo anticipo.
\end{document}